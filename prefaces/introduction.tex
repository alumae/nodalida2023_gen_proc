\setlength{\parskip}{0.3cm}%
\hyphenation{NoDaLiDa}

It is my great pleasure and honor to welcome you to the 24th Nordic Conference on Computational Linguistics (NoDaLiDa 2023)!

After a couple of years’ worth of conferences cancelled or held online (including the previous NoDaLiDa) we are extremely happy that NoDaLiDa 2023 is an onsite event. This is especially exciting given that for the first time in the history of NoDaLiDa conferences it takes place in Tórshavn, Faroe Islands.

The conference features three types of papers: long, short and demo papers. We are truly grateful to all the authors of papers submitted to this year’s conference, with 130 papers submitted, a more than 40\% increase over last year’s yield! In total, we accepted 79 papers: 49 long papers, 26 short papers and 4 demo papers. More than half of the accepted papers are student papers, in which the first author is a student (29 long, 17 short and 2 demo papers). We would like to thank the 113 members of the program committee who reviewed the papers for their contributions!

The 79 accepted papers are grouped into 12 oral and 2 poster sessions. In addition to these regular sessions the conference program also includes three keynote talks. We would like to extend our gratitude to the keynote speakers for agreeing to present their work at NoDaLiDa. Georg Rehm from DFKI will talk on the topic of ``Towards Digital Language Equality in Europe: An Overview of Recent Developments''. Hjalmar P. Petersen will talk about ``Aspects of the structure of Faroese''. Marta R. Costa-Jussà from Meta will talk about ``No-language-left-behind: Scaling Human-Centered Machine Translation and Toxicity at Scale''.

The main conference program is preceded by three workshops: NLP for Computer-Assisted Language Learning (NLP4CALL), the Constraint Grammar Workshop and Resources and representations for under-resourced languages and domains (RESOURCEFUL'2023). We thank the workshop organizers for their efforts and for expanding the main conference program with a focus on more specific research topics.

I would like to express sincere gratitude to the entire team behind organizing NoDaLiDa 2023. I was honored to receive the invitation to serve as the general chair from the NEALT board; thank you for trusting me with this role. My deepest gratitude goes to Tanel Alumäe for serving as the publications chair and his active participation, Inguna Skadiņa for serving as the workshop chair as well as Iben Nyholm Debess for serving as the main local chair and smoothly handling all associated aspects of conference organization. I also want to thank the rest of the program chairs, Lilja Øvrelid and Christian Hardmeier and the local co-chairs Bergur Djurhuus Hansen, Peter Juel Henrichsen and Sandra Saxov Lamhauge. Thank you everyone for your contributions, you are awesome!

NoDaLiDa 2023 received financial support from several institutions and we would like to thank them here: NEALT, Dictus, Málráðið, Tórshavnar kommuna, BankNordik, Digitaliseringsstyrelsen, University of the Faroe Islands, Nationella språkbanken, Elektron and Formula.

Welcome and enjoy the 24th Nordic Conference on Computational Linguistics!

Mark Fishel, General Chair

Tartu

May 2023

