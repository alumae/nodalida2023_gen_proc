Digital Language Equality (DLE) ``is the state of affairs in which all languages have the technological support and situational context necessary for them to continue to exist and to prosper as living languages in the digital age'', 
as we specified in one of our key reports of the EU project European Language Equality (ELE). 
Our empirical findings suggest that Europe is currently very far from having a situation in which all our languages are supported equally well through technologies. 
In this presentation, I'll give an overview of the two ELE projects and their main results and findings with a special focus on the Nordic languages 
(including insights from the FSTP projects supported through ELE2). This will also include a brief look back into the past, especially discussing the 
question if and where we have seen progress in the last, say, 15 years. Furthermore, I'll present an overview of our main strategic recommendations towards the European Union in terms of bringing about DLE in Europe by 2030.
The presentation will conclude with a look at other relevant activities in Europe, including, critically the Common European Language Data Space project, which started in early 2023.
