Phonological changes and later morphologization have led to different complex alternations in Faroese. These are argued to emerge especially in small languages, with little contact and tight networks. The alternations will be exemplified with 'skerping', palatalization, glide insertion and the quantity-shift. There will be a discussion of the morphology-phonology interface, where the suggestion is that Faroese has 3 strata, stem1, stem2 and a word- strata. Syntactic variation and different construction will be addressed and illustrated; in this context reflexives are included and the present reorganization of the case system of complements of prepositions, where speakers use semantic and structural case in a certain way. 
